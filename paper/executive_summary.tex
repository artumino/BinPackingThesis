% A LaTeX template for EXECUTIVE SUMMARY of the MSc Thesis submissions to 
% Politecnico di Milano (PoliMi) - School of Industrial and Information Engineering
%
% S. Bonetti, A. Gruttadauria, G. Mescolini, A. Zingaro
% e-mail: template-tesi-ingind@polimi.it
%
% Last Revision: October 2021
%
% Copyright 2021 Politecnico di Milano, Italy. NC-BY

\documentclass[11pt,a4paper,twocolumn]{article}

%------------------------------------------------------------------------------
%	REQUIRED PACKAGES AND  CONFIGURATIONS
%------------------------------------------------------------------------------
% PACKAGES FOR TITLES
\usepackage{titlesec}
\usepackage{color}

% PACKAGES FOR LANGUAGE AND FONT
\usepackage[utf8]{inputenc}
\usepackage[english]{babel}
\usepackage[T1]{fontenc} % Font encoding

% PACKAGES FOR IMAGES
\usepackage{graphicx}
\graphicspath{{Images/}} % Path for images' folder
\usepackage{eso-pic} % For the background picture on the title page
\usepackage{subfig} % Numbered and caption subfigures using \subfloat
\usepackage{caption} % Coloured captions
\usepackage{transparent}

% STANDARD MATH PACKAGES
\usepackage{amsmath}
\usepackage{amsthm}
\usepackage{bm}
\usepackage[overload]{empheq}  % For braced-style systems of equations

% PACKAGES FOR TABLES
\usepackage{tabularx}
\usepackage{longtable} % tables that can span several pages
\usepackage{colortbl}

% PACKAGES FOR ALGORITHMS (PSEUDO-CODE)
\usepackage{algorithm}
\usepackage{algorithmic}

% PACKAGES FOR REFERENCES & BIBLIOGRAPHY
\usepackage[colorlinks=true,linkcolor=black,anchorcolor=black,citecolor=black,filecolor=black,menucolor=black,runcolor=black,urlcolor=black]{hyperref} % Adds clickable links at references
\usepackage{cleveref}
\usepackage[authoryear, square, numbers, sort&compress]{natbib} % Square brackets, citing references with numbers, citations sorted by appearance in the text and compressed
\bibliographystyle{apalike} % You may use a different style adapted to your field

% PACKAGES FOR THE APPENDIX
\usepackage{appendix}

% PACKAGES FOR ITEMIZE & ENUMERATES 
\usepackage{enumitem}

% OTHER PACKAGES
\usepackage{amsthm,thmtools,xcolor} % Coloured "Theorem"
\usepackage{comment} % Comment part of code
\usepackage{fancyhdr} % Fancy headers and footers
\usepackage{lipsum} % Insert dummy text
\usepackage{tcolorbox} % Create coloured boxes (e.g. the one for the key-words)
\usepackage{stfloats} % Correct position of the tables

%-------------------------------------------------------------------------
%	NEW COMMANDS DEFINED
%-------------------------------------------------------------------------
% EXAMPLES OF NEW COMMANDS -> here you see how to define new commands
\newcommand{\bea}{\begin{eqnarray}} % Shortcut for equation arrays
\newcommand{\eea}{\end{eqnarray}}
\newcommand{\e}[1]{\times 10^{#1}}  % Powers of 10 notation
\newcommand{\mathbbm}[1]{\text{\usefont{U}{bbm}{m}{n}#1}} % From mathbbm.sty
\newcommand{\pdev}[2]{\frac{\partial#1}{\partial#2}}
% NB: you can also override some existing commands with the keyword \renewcommand

%----------------------------------------------------------------------------
%	ADD YOUR PACKAGES (be careful of package interaction)
%----------------------------------------------------------------------------


%----------------------------------------------------------------------------
%	ADD YOUR DEFINITIONS AND COMMANDS (be careful of existing commands)
%----------------------------------------------------------------------------


% Do not change Configuration_files/config.tex file unless you really know what you are doing. 
% This file ends the configuration procedures (e.g. customizing commands, definition of new commands)
\input{Executive_Configuration_files/config}

% Insert here the info that will be displayed into your Title page 
% -> title of your work
\renewcommand{\title}{Three-dimensional bin packing with vertical support}
% -> author name and surname
\renewcommand{\author}{Jacopo Libè}
% -> MSc course
\newcommand{\course}{Computer Science and Engineering - Ingegneria Informatica}
% -> advisor name and surname
\newcommand{\advisor}{Prof. Ola Jabali}
% IF AND ONLY IF you need to modify the co-supervisors you also have to modify the file Configuration_files/title_page.tex (ONLY where it is marked)
\newcommand{\firstcoadvisor}{Davide Croci} % insert if any otherwise comment
%\newcommand{\secondcoadvisor}{Name Surname} % insert if any otherwise comment
% -> academic year
\newcommand{\YEAR}{2021-2022}

%-------------------------------------------------------------------------
%	BEGIN OF YOUR DOCUMENT
%-------------------------------------------------------------------------
\begin{document}

%-----------------------------------------------------------------------------
% TITLE PAGE
%-----------------------------------------------------------------------------
% Do not change Configuration_files/TitlePage.tex (Modify it IF AND ONLY IF you need to add or delete the Co-advisors)
% This file creates the Title Page of the document
\input{Executive_Configuration_files/title_page}

%%%%%%%%%%%%%%%%%%%%%%%%%%%%%%
%%     THESIS MAIN TEXT     %%
%%%%%%%%%%%%%%%%%%%%%%%%%%%%%%



%% \section{Guidelines}
%% \label{sec:guidelines}
%% 
%% The Executive Summary is a critical overview of your thesis
%% with a focus on the main achievements that have emerged from your research.
%% 
%% The Executive Summary should be organized in sections/paragraphs
%% in order to better highlight the major points of your work.
%% The length should range from four to six pages depending on the length of the thesis manuscript.
%% Keep the Executive Summary concise enough to be effective but long enough to allow it to be complete.
%% It should be written after completing the thesis manuscript as a stand-alone independent document
%% of sufficient clarity and detail to ensure that the reader can figure out the overall objectives,
%% the methodology employed and the results/impact of your research.
%% 
%% In writing the Executive Summary, keep in mind that it is not an abstract, it is not a preface,
%% and it is not a random collection of highlights.
%% With a few exceptions, do not simply cut and paste whole sections or paragraphs of the thesis manuscript
%% into a disorganized and cluttered Executive Summary.
%% You should reorganize information to be informative as well as concise.
%% 
%% The Executive Summary could contain a few important equations related to your work.
%% It could also include the most relevant figures and tables taken or elaborated from the thesis manuscript.
%% 
%% You should also include in the Executive Summary the very essential bibliography of your study.
%% The number of selected references should range from three to five depending on the type of work.
%% 
%% The Executive Summary should contain a final section reporting the main conclusions drawn from your research.

%-----------------------------------------------------------------------------
% INTRODUCTION
%-----------------------------------------------------------------------------
\section{Introduction}
\label{sec:introduction}

Recent progress in the digitalization of industrial processes led to a rise in studies on the Three-Dimensional Bin Packing Problem (3D-BPP).
The problem consists in packing a set of items in the minimum number of bins without any overlap.
When considering real-world settings, the addition of new practical constraints is required.
Previous studies in other fields related to container loading and pallet loading have shown that static stability of the bins is a crucial aspect to consider (\cite{BORTFELDT20131}).
In this thesis, we address a version of the bin packing problem stemming from a real case study of mixed-case palletization: the Three-Dimensional Bin Packing Problem with Vertical Support (3D-BPPVS).
We extend the standard formulation of the bin packing problem by ensuring that all items that are packed inside a bin will not fall, and we refer to this property as the vertical support.

Our research stems from the case study of a logistics company in northern Italy.
The company manages large warehouses where automated lines bring boxes to different packing stations, and then they are loaded onto pallets of standard size.
Since the company is dealing directly with customers' orders, boxes have very different sizes and are usually packed in smaller quantities.
Moreover, the assortment of items to pack is strongly heterogeneous which makes the use of layered approaches to have sub-optimal results.
During the palletization, the lower levels of already packed items are wrapped to ensure better overall stability of the pallet.
This wrapping procedure requires that the amount of unused space between items is minimal. The company measures this property with a metric called cage ratio.
Cage ratio is the ratio between the volume of the packed items inside a bin and the volume of the cuboid which surrounds them, the cage.
The cage has the same base as the bin and height equal to the highest packed item inside the bin.
Current commercial solutions employed by the company have solutions with around $60\%$ cage ratio, and a target of $70\%$ was set as a benchmark for our work.

\section{Gap Identification}
The 3D-BPP is the generalization of the one-dimensional bin packing problem which is NP-Hard (\cite{martello2000three}).
Exact methods can only solve small instances of the problem which means that most solutions proposed in the literature are heuristics.
The concept of vertical support received most of its contribution from the literature of Container Loading Problems (CLP) and Pallet Loading Problems (PLP).
As noted in \cite{BORTFELDT20131}, static stability is usually implicitly enforced as a consequence of load compactness, or explicitly guaranteed by using filler material in a postprocessing step.
Most heuristics for CLPS and PLPs try to build dense layers composed of similar items that they then stack, reducing the problem to a one-dimensional bin packing problem.
Layers are filtered based on the fill-rate and when they are below a certain threshold they are discarded (e.g., \cite{elhedhli2019three, Alonso2020}).
This means that when no new layer can be built, new bins are opened, simpler placement methods are used to pack the remaining items or filler material is used to complete the layers.

Our solution to the problem fills the gap in the research by finding solutions to the 3D-BPPVS without explicitly building layers, and without the use of filler material.

\section{Proposed Solution}


\section{Results}

\begin{table}[htbp]
    \caption{Average execution time of literature results with bin gap}
    \centering
    \resizebox{\textwidth}{!}{\begin{tabular}{|l|l|c|c|c|c|c|}
    \hline
    \multicolumn{2}{|c|}{\textbf{Heuristic}} & \multicolumn{4}{|c|}{\textbf{Execution Time (s)}} & \textbf{Bin Gap \%} \\ \hline
    \multicolumn{2}{|c|}{} & n=50 & n=100 & n=150 & n=200 &  \\ \hline
    \textbf{Group By Hash} & k=1 & 0.03 & 0.11 & 0.28 & 0.54 & 5.82 \\ 
     & k=5 & 0.08 & 0.38 & 1.00 & 2.09 & 5.56 \\ 
     & k=10 & 0.15 & 0.73 & 1.93 & 4.00 & 5.54 \\ 
     & k=20 & 0.29 & 1.40 & 3.77 & 7.71 & 5.30 \\ 
     & k=50 & 0.70 & 3.50 & 9.39 & 19.59 & 5.19 \\ \hline
    \textbf{Single Placement} & k=1 & 0.05 & 0.18 & 0.50 & 1.05 & 5.61 \\ 
     & k=5 & 0.12 & 0.72 & 2.10 & 4.62 & 5.26 \\ 
     & k=10 & 0.23 & 1.38 & 4.11 & 8.95 & 5.19 \\ 
     & k=20 & 0.46 & 2.67 & 8.21 & 17.64 & 4.98 \\ 
     & k=50 & 1.12 & 6.45 & 20.39 & 43.50 & 4.75 \\ \hline
     \hline
     \multicolumn{2}{|l|}{\textbf{BRKGA-VD}} & 1.85 & 8.69 & 20.53 & 39.85 & 0.00 \\ \hline
    \end{tabular}}
    \label{exp:literature_time_gap}
\end{table}
    

%-----------------------------------------------------------------------------
% CONCLUSION
%-----------------------------------------------------------------------------
\section{Conclusions}
A final section containing the main conclusions of your research/study have to be inserted here.

%---------------------------------------------------------------------------
%  ACKNOWLEDGEMENTS 
%---------------------------------------------------------------------------
\section{Acknowledgements}
Here you might want to acknowledge someone.

%---------------------------------------------------------------------------
%  BIBLIOGRAPHY
%---------------------------------------------------------------------------
% Remember to insert here only the essential bibliography of your work
\setcitestyle{numbers} % set the citation style to ``numbers''.
\bibliography{literature.bib} % automatically inserted and ordered with this command 

\end{document}