
% ABSTRACT IN ITALIAN
\chapter*{Abstract in lingua italiana}
I recenti progressi nella digitalizzazione dei processi industriali hanno portato a un aumento degli studi sul problema dell'imballaggio tridimensionale dei contenitori.
Il problema consiste nell'impacchettare un insieme di articoli nel numero minimo di contenitori senza alcuna sovrapposizione.
Quando si considerano istanze reali del problema, è necessaria l'aggiunta di nuovi vincoli pratici.
Studi precedenti in altri campi relativi al carico di container e pallet hanno dimostrato che la stabilità statica dei contenitori è un aspetto cruciale da considerare.
La maggior parte delle soluzioni in letteratura affronta il problema del supporto verticale implicitamente generando strati densi di articoli che vengono poi impilati per riempire un contenitore.
\\
In questa tesi, formuliamo un modello mixed-integer linear programming per il problema dell'imballaggio tridimensionale dei contenitori con una versione discretizzata del vincolo del supporto verticale.
Proponiamo poi un'euristica costruttiva che riempie i contenitori senza costruire esplicitamente soluzioni a strati o usare materiale di riempimento, come è solito in soluzioni dalla letteratura.
Inoltre, modifichiamo l'algoritmo bidimensionale Extreme Points per considerare il supporto verticale.
Introduciamo, poi, un algoritmo di beam-search che valuta diversi posizionamenti fatti dalla nostra euristica costruttiva e filtra le soluzioni duplicate.
Forniamo anche un set di istanze con articoli campionati da una popolazione di prodotti reali.
Infine, convalidiamo i risultati del nostro algoritmo usando diversi set di dati, sia dalla letteratura che da un caso di studio di pallettizzazione mista.
\\
\\
\textbf{Parole chiave:} imballaggio tridimensionale, supporto verticale, stabilità statica, pallettizzazione mista % Keywords (italian)