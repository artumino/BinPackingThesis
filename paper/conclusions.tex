In this thesis, we studied the Three-Dimensional Single Bin-Size Bin Packing Problem With Support.
The first contribution of our work is a constructive heuristic that uses a modified version of the first-fit two-dimensional Extreme Points algorithm of \citet{crainic2008extreme}. This new version of the Extreme Points algorithm is able to consider the constraints of area and vertex support.
Our heuristic builds solutions to the single bin 3D-BPP by filling planes called support planes, generated based on the items inserted previously.
We evaluate insertions in two different placement modes, allowing for placements of groups of similar objects.

We also propose a beam-search algorithm that uses multiple instances of our constructive heuristic and evaluates different sequences of insertions. We developed a hashing mechanism to avoid considering duplicate solutions in the process.

We validated our heuristic against small instances solved with our MILP formulation, and we obtained results that were on par with the model in significantly smaller computational time.
We then compared the results of a relaxed version of our heuristic against other heuristics from the literature on classical benchmark instances from \citet{martello2000three}. Here we detected an average gap of $5.32\%$ against the best solutions provided by other heuristics, however we were able to solve the same problem in a fraction of their computational time. We consider this as a great result since it states that our algorithm is competitive also in the realm of 3D-SBSBPP without support.
Finally, we generated a data set of problem instances based on real-world products from our case study, and we used them to evaluate our heuristic. In most configurations, our solutions exceeded the target metric of $70\%$ cage ratio, with some configurations having a negligible execution time.

Further research could introduce new practical constraints considered in the literature like family groupings, load-bearing, and compatibilities between items.
The scoring function used to sort states can also be improved to avoid lexicographic ordering, which in some problem instances can cause a temporary worsening of solutions.
As an alternative to changing the scoring mechanism, some backtracking could be added to the beam-search algorithm.
Moreover, our extreme points variant could be extended with the same projection logic of the standard formulation of \citet{crainic2008extreme}, while adding special cuboids accounting for empty spaces in the set of supporting items.
Finally, improvement heuristics could be adapted to account for the support constraint like, for example, space defragmentation techniques introduced by \cite{ZHU2012452}.