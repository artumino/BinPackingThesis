The three-dimensional bin packing problem (3D-BPP) involves packing, without any overlap, a set of small items into the minimum number of bins.
In its most standard form, each item $i$ is a cuboid of dimensions $(w_i, d_i, h_i)$ and each bin is a cuboid with fixed dimensions $(W, D, H)$.
Items can only be placed with edges that are parallel to the sides of the bin and can be rotated by $90$ degrees along their vertical axis.

The 3D-BPP is part of the family of Cutting and Packing problems, where a set of small items (boxes) needs to be packed inside a set of large ones (bins).
Based on the dimensionality of the problem and on the number of items with different shapes, \cite{WASCHER20071109} identifier proposed typology of cutting and packing problems. Versions of the 3D-BPP with strongly heterogeneous items are classified as three-dimensional single bin-size bin packing problems (3D-SBSBPP). 

In this thesis, we address a variation of the 3D-SBSBPP stemming from a real case study of mixed-case palletization: the Three-Dimensional Bin Packing Problem With Vertical Support (3D-BPPWS).
Modeling real-world case studies with the 3D-SBSBPP requires additional constraints to be considered. We extend the standard formulation of the 3D-SBSBPP by ensuring that all items that are packed inside a bin will not fall, and we refer to this property as the vertical support constraint.
Support constraints are usually defined based on the amount of area of an item that lies on top of other items, or by the number of corners of an item that rest on top of other items (\cite{GZARA20201062, paquay2016mixed, kurpel2020exact}).

The standard 3D-SBSBPP is strongly NP-hard since it is a generalization of the one-dimensional bin packing problem \citep{martello2000three}.
Since our problem is a generalization of the 3D-SBSBPP, exact solution algorithms are only able to solve small instances of the problem, therefore heuristic approaches need to be used to solve larger instances.
Heuristics designed to solve the 3D-SBSBPP do not address the concept of static stability and allow solutions with unsupported items.
The concept of static stability has received most of its contributions in Pallet Loading Problems and Container Loading Problems (\cite{Calzavara2021, kurpel2020exact}).
In these publications, the concept of support is addressed explicitly by building layers or walls of items with high density, which allows them to reduce the problem to a one-dimensional packing problem (\cite{BORTFELDT20131}).
In mixed-case palletization scenarios, layer-based solutions represent the majority of work in the literature. They usually work by stacking layers ordered by density until the density of the last generated layers falls below a certain threshold.
Once no more layers can be generated, simpler techniques are employed to pack the remaining items (\cite{elhedhli2019three}), or the use of filler boxes is employed to increase the layer's density (\cite{Calzavara2021}).

In this thesis, we approach the constraint of support by developing a constructive heuristic that fills bins without explicitly building layered solutions or using filler boxes.
We then introduce a beam search heuristic that expands the constructive heuristic's solution space by exploring different orders of item palletization.
We also provide a formulation for the 3D-BPPWS with a discretized version of the support constraint. Such a formulation is used to validate our heuristic on small instances of the problem.
Finally, we provide a generated data set based on real-world instances that we use to benchmark our heuristic.

\section{Case study}
\label{sec:intro:case_study}%
The work of this thesis stems from the case study of a logistics company in northern Italy.
The company manages large warehouses where automated lines bring boxes to different packing stations, and then they are loaded onto pallets of standard size.
Each box is loaded manually by an operator, and soon as the height of the pallet reaches a certain threshold, the packing station lowers it and wraps it with an elastic material that guarantees its stability.
This wrapping improves the stability of the pallet while boxes are still being loaded on the top. To avoid uneven surfaces during the wrapping phase, pallets should not have empty spaces inside.
When dealing directly with customers' orders, boxes have very different sizes and they are usually packed in smaller quantities. This implies that layer-based pallet loading solutions are impractical in these cases, since it is usually not possible to build full layers of boxes of the same height.
The company is interested in building pallets that do not have empty spaces inside, and they measure this property with a metric called cage ratio. A formal definition of this metric is reported in \cref{eq:cage_ratio}.
To increase the efficiency of the wrapping and to allow for the stacking of pallets, solutions with a high cage ratio are required.
The cage ratio of commercial solutions currently implemented by the company is around $60\%$, and a target cage ratio for our case study was set at $70\%$.

\newpage
\section{Overview}
\label{sec:intro:overview}%
In \cref{chapter:literature} we review the relevant literature on the three-dimensional bin packing problem and the cutting and packing problems dealing with vertical support.
In \cref{chapter:problem} we give a formal definition of the problem and formulate a mixed-integer linear programming model that we will use to validate the proposed solution algorithm.
Since the model can not be used to solve real-world instances, in \cref{chapter:heuristics} we propose a heuristic algorithm that is able to solve larger problem instances.
In chapter \cref{chapther:experiments} we present our computational experiments. We compare our heuristic algorithm against relevant heuristics from the literature, against the solutions from our MILP model, and against solutions from our real-world case study. We also describe the process that we used to generate new instances.
Finally, in \cref{chapther:conclusions} we give final remarks and list possible further developments of this research.