\chapter{Solution algorithms}\label{chapter:heuristics}
\section{Beam Search}\label{chapter:heuristics:beamsearch}
Beam Search (BS) %TODO: maybe add info on literature
is an heuristic graph search algorithm designed for systems with limited memory where expanding every possible node is unfeasible.
The idea behind BS is to conduct a iterative truncated breadth-first search where, at each iteration, expanded nodes are ranked based on an heuristic and only the best ones are further explored.
To perform BS one must define the node structure, an expansion function to generate new nodes from an existing one, an evaluation function to compare nodes between eachother and a function to determine if a node is a solution to the problem.
Let $s_i$ be a node in the graph of possible solutions of the 3DBPP, %TODO 3DBPP o 3DSPP?
$s_i$ can be seen as an instance of the problem where a sequence of placements has taken place.
An expansion of a node $s_i$ generates a new node $s_j$ where a placement has occured for a given set of items.
Since the expansion function would generate a combinatorial number of nodes, each node will be expanded based on an expansion heuristic designed for the problem.
The described procedure is rappresented by algorithm \ref{algo:beamsearch}.

\begin{algorithm}[H] \label{algo:beamsearch}
    \DontPrintSemicolon
    \SetAlgoLined
    \SetKwInOut{Input}{input}
    \SetKwInOut{Output}{output}
    \Input{$S^0, k$}
    \Output{$s_{best}$}
    $S^t \gets S^0$\;
    $S_{final} \gets \emptyset$\;
    \Repeat{$S^t \neq \emptyset $}{
        $S^{t+1} \gets Expand(S^t)$ (algo. \ref{algo:state_successor})\;
        $S_{final} \gets S_{final} \cup \{s \in S^{t+1} : IsFinal(s)\}$ (def. \ref{def:state_final})\;
        $S^{t+1} \gets S^{t+1} \setminus S_{final}$\;
        $S^{t+1} \gets Sort(S^{t+1})$ (sec. \ref{ssec:scoring_states})\;
        $S^t \gets \emptyset$\;
        $i \gets 0$\;
        $seen \gets \emptyset$\;
        \ForAll{$s \in S^{t+1}$}{
            \If{$hash^s \in seen$}{
                continue\;
            }
            $S^t \gets S^t \cup Commit(s)$ (algo. \ref{algo:state_commit})\;
            $seen \gets seen \cup \{ hash^s \}$\;
            $i \gets i+1$\;
            \If{$i > k$}{
                break\;
            }
        }
    }
    $S_{final} \gets Sort(S_{final})$\;
    \Return{best element of $S_{final}$}
    \caption{Beam search}
\end{algorithm}


\begin{algorithm}[H] \label{algo:state_successor}
    \DontPrintSemicolon
    \SetAlgoLined
    \SetKwInOut{Input}{input}
    \SetKwInOut{Output}{output}
    \Input{$S$}
    \Output{$S_{new}$}
    \ForAll{$s \in S$}{
        $S_{new} \leftarrow \emptyset$\;
        $I_{family} \leftarrow GroupByFamily(s.Unpacked)$\;
        %$minPlacement \leftarrow \{(b, 0) | \forall b \in s.Bins \}$\;
        $placed \leftarrow false$\;
        \ForAll{$(family, I) \in I_{family}$}{
            \ForAll{$bin \in s.Bins$}{
                $placement \leftarrow SPBestInsertion(s, bin, I)$ (Algorithm \ref{algo:sp_bestinsertion})\;
                \If{$placement \neq \emptyset$}{
                    $placed \leftarrow true$\;
                    %$minPlacement(bin) \leftarrow min(minPlacement(bin), placement.z)$\;
                    %TODO: abuso di notazione
                    $S_{new} \leftarrow S_{new} \cup Next(s, placement)$\;
                }
            }
        }
        \If{$placed = false$}{
            $S_{new} \leftarrow S_{new} \cup OpenNewBin(s)$\;
        }
        %TODO: Filter?
    }
    \Return{$S_{new}$}
    \caption{Expand}
\end{algorithm}

\subsection{Scoring States}\label{chapter:heuristics:beamsearch:scoring}
In order to sort nodes, a scoring function needs to be defined over the nodes. 
To allow the BS to explore better solutions the scoring function can't be as flat as the objective function defined in the mathematical formulation of the problem. %TODO aggiungere ref

\section{Support Plane}\label{chapter:heuristics:supportplane}
Support Plane (SP) is an heuristic introduced in this thesis based on an underlying 2DBPP heuristic which is used to evaluate feasible expansions of a given node in the BS.
The proposed heuristic ensures that the constraint of support isn't violated.
The idea at the base of SP is to build a solution to the 3DSPP by filling 2D planes called support planes.
Each support plane can be characterized by the triple $S_z = (z, I_{support}, I_{upper})$ where $z$ is the height of the plane, $I_{support}$ is the set of the items that can offer support to items placed on the plane %TODO link to support
and $I_{upper}$ is the set of items that will be obstacles to potential new items placed on it.

\section{AABB Tree}\label{chapter:heuristics:aabbtree}

\section{Max Rects}\label{chapter:heuristics:maxrects}