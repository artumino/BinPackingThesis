In this thesis, we studied the problem of three-dimensional single bin-size bin packing with support by providing two heuristics. 
The first contribution is a constructive heuristic that uses a modified version of the first-fit two-dimensional extreme points heuristic of \citet{crainic2008extreme} that includes the notion of area and vertex support.
The heuristic builds solutions to the single bin 3D-BPP by filling planes called support planes generated based on the inserted items; these structures are then used to facilitate support constraint calculations. 
Although we do not explicitly build layers, we evaluate insertions in two different placement modes, allowing for placements of groups of similar objects.
We also propose a beam-search algorithm as an expansion step. It uses multiple instances of our constructive heuristic and evaluates different sequences of insertions with a hashing mechanism to avoid considering insertions that lead to the same solutions.

We validated our heuristic against small instances solved with our MILP formulation, where we got results that were on par with the model in significantly less computation time.
We then compared the results of a relaxed version of our heuristic against other heuristics from the literature on classical benchmark instances from \citet{martello2000three} where we detected an average gap of $5.32\%$ against the best values across all heuristics.
Finally, we generated a data set of problem instances based on real-world products from our case study, which we used to evaluate our heuristic. In most configurations, the heuristic exceeded the target metric of $70\%$ cage ratio, with some configurations having a negligible execution time.

Further research could introduce new practical constraints considered in the literature like family groupings, load-bearing, and compatibilities between items.
The scoring function used to sort states can also be improved to avoid lexicographic ordering, which in some problem instances caused a temporary worsening of solutions in our computational tests.
As an alternative to changing the scoring mechanism, some backtracking could be added to the beam-search algorithm.
Our extreme points variant could be extended with the same projection logic of the standard formulation of \citet{crainic2008extreme} while adding special cuboids accounting for empty spaces in the set of supporting items.
Finally, improvement heuristics could be adapted to account for the support constraint like, for example, space defragmentation techniques introduced by \cite{ZHU2012452}.